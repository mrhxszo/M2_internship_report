\section{Introduction}

\medskip

The continuous miniaturization of microelectronic components, driven by Moore's Law, 
has led to a significant reduction in transistor size and increased chip complexity.
This rapid advancement has presented new challenges in the field of metrology, the science of measurement.
Existing metrology techniques, such as Optical Critical Dimension (OCD) and Critical Dimension Scanning 
Electron Microscopy (CDSEM), are reaching their limits in terms of resolution and accuracy as feature sizes
shrink to the nanometer scale.


\begin{figure}[h]
    \centering
    \includegraphics[width=0.7\textwidth]{images/moore's_law.png}
    \caption{Evolution of microelectronics and illustration of the need for advanced metrology techniques \cite{moore_law}.}
    \label{fig:evolution_microelectronics}
\end{figure}

\FloatBarrier

To address these challenges, a new metrology technique called Critical Dimension Small Angle X-ray Scattering
(CD-SAXS) is being developed. In the context of microelectronics, CD refers to caracterizing feature length
of the nanostructures. CDs are the measurements that directly affect the performance and function of the devices.
For example in the figure \ref{fig:cd_example}, the CD is the overlay gap between two layers of nanostructures.
This length is very important to control in order to have a good conduction between the two layers, hence
a good performance of the device.

\begin{figure}[h]
    \centering
    \includegraphics[width=0.4\textwidth]{images/CD_example.png}
    \caption{Illsuration of what a CD measurement, in this case CD is the overlay gap between two layers of nanostructures.}
    \label{fig:cd_example}
\end{figure}


CD-SAXS utilizes short-wavelength X-rays $(\lambda \approx 0.05 - 5 nm)$, to probe the internal structure of
materials, providing high-resolution measurements of CD with greater accuracy than
conventional methods. CEA-Leti, a leading research institute in microelectronics, is actively involved in
the development of CD-SAXS technology.

\medskip

This work-study project focused on the development of a coherent software for the fit and analysis of CD-SAXS
data. The software aims to streamline the data processing workflow and enhance the accuracy of CD measurements.
The project involved a comprehensive understanding of CD-SAXS theory, data collection procedures, and fitting
algorithms.

\medskip

The report begins with an overview of the context of the project, highlighting the evolution of
microelectronics and the need for advanced metrology techniques. It then delves into the CD-SAXS technique, 
explaining the principles, data collection, fitting, and analysis. Then the subsequent
section describes the software development process, outlining the software's functionalities and design. Finally, the report concludes with a summary of the project's achievements and
outlines potential future directions.