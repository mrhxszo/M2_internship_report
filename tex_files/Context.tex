\section{Context}
\subsection{Host organisation}

\medskip

The French Alternative Energies and Atomic Energy Commission (CEA) stands as a cornerstone
of the nation's research landscape. Its multifaceted expertise encompasses a broad spectrum
of fields, including nuclear energy, renewable energy, technological research for industry
, material sciences, health and life sciences, and defense and security. The CEA's network
of research centers spans across France, each with its unique specializations and areas
of excellence. Among these, CEA Grenoble holds a prominent position, where I have the 
privilege of pursuing my work-study program. I am part of the Leti Institute,
a research center dedicated to microelectronics and nanotechnologies. More specifically, 
I was with “Materials and Structures Properties Laboratory” (MSPL) which is under “Technology Platforms 
Department” (TPFD), one of six different departments of Leti.

\medskip

\subsubsection{History}

\medskip

The French atomic energy commission, CEA, was born in 1945 after World War II.  Its mission was to develop nuclear 
expertise for France.  Pioneering scientists like Frédéric Joliot-Curie and Francis Perrin led the way in building 
research reactors and nuclear power plants.

\medskip

The CEA didn't stop at just nuclear energy.  In the 1960s, they began to diversify into new areas like renewable energy,
 micro and nanotechnologies, defense, and healthcare.  This diversification led to the creation of specialized research
  centers, including the future innovation hub, CEA Grenoble.

\medskip

CEA Grenoble was founded in 1956 by Nobel laureate physicist Louis Néel.  He saw the scientific potential of the Grenoble
region and his vision proved to be true.  The center grew rapidly in the following decades, attracting talent and investment
from around the world.

\medskip

It came to be known as France's "atomic capital" due to its research reactors.  However, their influence went far beyond
 nuclear.  They developed their first integrated circuit in 1965, launching their journey into micro and nanotechnologies.  
 They also played a key role in creating Minatec, the first European hub for excellence in this field.  In addition, they 
 became a leader in renewable energy research with the Institut national de l'énergie solaire (Ines).

\medskip

Today, CEA Grenoble is a research powerhouse with over 2,500 researchers and technicians.  Their campus houses specialized
 institutes in various fields, from healthcare to digital technologies.  It's also the headquarters for CEA Tech, the
  technological branch of the CEA with over 4,500 researchers across France.

\medskip

\subsubsection{Sectors of activity}

CEA Grenoble plays a pivotal role in the nation's economic and technological 
advancement through its groundbreaking research and innovations across diverse fields.

\begin{itemize}
  \item \textbf{Energy and Sustainability}: Supporting current and future nuclear power, exploring solar, hydrogen for carbon
   neutrality (2050), researching SMRs and thermonuclear fusion.
  \item \textbf{Digital Technologies}: Contributing through the SPIN program for spintronics (frugal, agile, sustainable computing)
   aligned with France 2030 plan.
  \item \textbf{Healthcare}: Distinguished research in biology and biotechnology for health, addressing current and future challenges.
   (e.g., Laboratoire de biologie et biotechnologie pour la santé)
  \item \textbf{Defense}: Traditionally significant role, developing cutting-edge technologies for national security and defense 
  (less documented for Grenoble).
\end{itemize}

CEA Liten for example also serves as an innovation hub for new energy technologies and nanomaterials, emphasizing energy diversification
 and renewable energy integration. Their research encompasses solar photovoltaics, energy storage, and transportation (hydrogen and fuel
  cells).


\subsubsection{Future and oreintation strategy}

\medskip

The French Alternative Energies and Atomic Energy Commission (CEA) stands as a powerhouse for innovation in France.
  Spanning energy, healthcare, defense, and digital technologies, the CEA pushes boundaries by collaborating with universities
  and industry on ambitious R\&D projects.

\medskip

Their focus is clear: develop transformative solutions for global challenges. 
 This includes renewable energy sources, innovative healthcare systems, advanced defense solutions, 
 and disruptive digital technologies.  Sustainability is paramount, with research prioritizing energy 
 efficiency and minimizing environmental impact.

\medskip

The CEA partners with leading institutions to accelerate technology transfer and create open innovation ecosystems. 
 These partnerships combine expertise, resources, and networks, leading to breakthrough technologies with significant 
 economic and social value.

\medskip

CEA Grenoble exemplifies this innovative spirit. They focus on cutting-edge technologies like AI, advanced materials,
 nanotechnologies, and quantum technologies.  Their research aims to revolutionize industries and improve lives, from
 developing next-generation batteries to creating innovative medical devices and advancing digital technologies.

\medskip

The challenges they tackle are vast: climate change, the energy transition, emerging diseases, cybersecurity, and 
technological sovereignty.  The CEA goes beyond just solutions; they strive to influence public policy and raise awareness. 
 Their research is guided by a long-term vision, anticipating future challenges and preparing for them through innovation.

\medskip

The CEA is a vital force in shaping a sustainable future.  Their commitment to innovation promises clean energy sources, 
advanced healthcare, robust national security, and a cutting-edge digital landscape.  By working collaboratively and addressing
 global challenges head-on, the CEA positions itself as a leader in the global scientific and technological landscape.

\medskip

\subsection{Project description}
\subsection{Objectives}